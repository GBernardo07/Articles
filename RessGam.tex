\documentclass[12pt, oneside,twocolumns, a4paper, brazil]{abntex2}

\usepackage{lipsum}
\usepackage{lmodern}					
\usepackage[T1]{fontenc}		
\usepackage[utf8]{inputenc}
\usepackage[brazil]{babel}
\usepackage{indentfirst}
\usepackage{hyperref}
\usepackage[num, overcite]{abntex2cite}

\citebrackets()
\hypersetup{hidelinks}
% \setlength{\columnseprule}{0.4pt}

\title{Ressignificação de posicionamento}
\date{2022}
\author{Gabriel Bernardo}

\makeatletter
\setlength{\@fptop}{5pt} 
\makeatother

\setlength{\parindent}{1.3cm}
\setlength{\parskip}{0.2cm}

\begin{document}

\selectlanguage{brazil}
\frenchspacing

\maketitle

\textual

A capacidade humana de produzir conhecimento é profundamente limitada pela condição de cegueira inerente à maioria das pessoas, frente a uma variedade de aspectos da existência. Essa cegueira é direcionada não só a certas dimensões de realidade - cuja existência é constatada por algumas tantas desafortunadas subjetividades - mas também à percepção da instabilidade das convicções e certezas que carregam consigo. \par
É fundamentado nessa noção que apresento algumas reflexões em torno do que Rabban Gamaliel, ao tempo do primeiro século depois de Cristo, argumentou em defesa dos primeiros seguidores de Jesus de Nazaré frente ao Sinédrio. A Revelação Cósmica, assunto do qual tratarei com razoável superficialidade, fornece, ainda, conceitos e painéis que permitem um diálogo entre esse ocorrido e o atual contexto das coisas.

\section*{Quem foi Gamaliel}

A Federação Espírita do Paraná (FEP) produziu um artigo\cite{artFEP} muito proveitoso para o estudo dessa figura. Lá são apresentadas a sua origem e sua atuação como doutor da lei judaica.

\begin{citacao}
Gamaliel pertencia à seita dos fariseus, que tinha nas mãos, à época primeira da Boa Nova, praticamente todo o alto ensino religioso. Era dos homens mais notáveis da seita farisaica. Representava, como o grande Hilel, a tendência liberal. (...)
\end{citacao}

Pode-se imaginar, então, a influência que essa personalidade foi capaz de exercer em sua época, bem como o impacto de seus ensinamentos nas gerações de fariseus que sucederam-no. Importa destacar, porém, o cenário que se desenhou ao seu redor após a crucificação de Jesus.\par
Segundo o Novo Testamento, Pedro e João, após terem sido libertados da prisão, apresentaram-se no templo de Jerusalém para serem julgados por comportamentos contrários aos interesses do Sinédrio. Após serem sentenciados à morte, Gamaliel ordenou que os retirassem do recinto para argumentar da seguinte forma: 

\begin{citacao}

E disse-lhes: Homens israelitas, acautelai-vos a respeito do que haveis de fazer a estes homens,
Porque antes destes dias levantou-se Teudas, dizendo ser alguém; a este se ajuntou o número de uns quatrocentos homens; o qual foi morto, e todos os que lhe deram ouvidos foram dispersos e reduzidos a nada.
Depois deste levantou-se Judas, o galileu, nos dias do alistamento, e levou muito povo após si; mas também este pereceu, e todos os que lhe deram ouvidos foram dispersos.
E agora digo-vos: \textbf{Dai de mão a estes homens, e deixai-os, porque, se este conselho ou esta obra é de homens, se desfará,
Mas, se é de Deus, não podereis desfazê-la; para que não aconteça serdes também achados combatendo contra Deus.} (Grifo meu) (Atos 5:35-39)

\end{citacao}

Essa colocação foi suficiente para comover seus companheiros e absolver os jovens. Esse foi um dos maiores legados que o Rabban deixou para a sociedade contemporânea: a noção de que aquilo que é de Deus se sustenta independente do esforço humano no sentido contrário. A tolerância característica do seu temperamento possibilitou tal compreensão do caso e garantiu sua canonização pelas igrejas cristãs mais tarde. \par

\section*{Uma nova perspectiva}

Em 2009, Leonardo Gonçalves expôs seu posicionamento a respeito de alguns discursos recorrentes em seu meio. Dirigentes de associações religiosas levantavam a voz - supostamente respaldados pelo posicionamento de Gamaliel - para dizer que se suas obras fossem ilegítimas, pereceriam frente a Deus, mas se fossem dEle, ninguém os pararia. Leonardo argumenta que a hermenêutica bíblica razoavelmente posta apenas descreve o discurso do antigo fariseu, não o prescreve como correto ou louvável. \par
Assim, chama a atenção para a quantidade de seitas religiosas que, sendo ainda mais antigas que o cristianismo, aumentam cada vez mais seu número de adeptos, dizendo: "Ora, se o argumento de Gamaliel estiver correto, então serei forçado a crer que o budismo, religião que ensina a reencarnação, animismo e tantas outras abstrações, também é de Deus!"\cite{art1}. Com um tom de notória revolta, deixa claro seu ponto de vista: nem tudo o que resiste ao tempos é obra de Deus. \par
É claro que Gonçalves, sendo único portador da coroa do bom senso e da razoabilidade, assume que a sua religião é a que possui o carimbo da verdade. Tudo o que for dissidente, então, pode ser obra de qualquer fonte, menos da vontade de Deus - o responsável por credibilizar as ideias com selos de "verdadeiro" e "falso". Como poderia essa variedade de doutrinas serem todas adequadas para descrever a realidade ao mesmo tempo? Não é possível, alguma deve estar certa e todas as outras erradas. Mas se Gamaliel estivesse com a razão, tudo o que chegou até hoje como conhecimento religioso aceito abertamente por massas de pessoas estaria igualmente correto. \par
Essa visão é problemática por um simples motivo: não existe nada certo. A humanidade sempre que buscou atingir algum nível de certeza sobre as coisas acabou percebendo, mais tarde, que não era muito bem como se imaginava. Século atrás de século vamos desenvolvendo novos sistemas de pensamento e metodologias científicas na tentativa de esclarecer os fenômenos da forma mais adequada possível, com maior ou menor nível de precisão - e não parece que Deus está tendo um papel muito relevante nessa empreitada. \par
Tirando-O da equação, vamos substituir no discurso de Gamaliel (atualizando-o, de certo modo) a atuação de Deus pelo estatuto de verdade. Assim, ficaríamos com algo como: "se essa proposição não é verdade, se desfará; mas se é verdade, não podereis desfazê-la". Podemos observar, no entanto, que no curso da história humana só o que se observa são antigas verdades sendo atropeladas por novas verdades. Não houve um momento sequer em que não se descobrisse ou propusesse algo novo sem desconstruir minimamente o que já existia. Parece, então, que as nossas verdades carregam consigo a possibilidade de serem falseadas ou invalidadas. \par
É muito provável que o Rabban realmente não pudesse propor nada diferente daquilo, no fim das contas. De um jeito ou de outro, era fruto de sua época o pensamento de que se algo era obra de Deus (o soberano criador de todas as coisas), se manteria por toda a eternidade. Acreditava-se piamente que havia um ser preocupado com o futuro dos seus escolhidos e capaz de fazer valer sua vontade sobre a dos demais. O discurso descrito nas passagens bíblicas foi útil para salvar os discípulos e dar continuidade ao projeto de Jesus, mas aos dias atuais não serve de muita coisa. \par
Uma consequência direta do que Gamaliel propôs é a constatação de que podemos aferir a validade de uma premissa pela sua permanência e sustentabilidade na história (caso desapareça, é porque não é verdade). Acontece que um erro propagado através das gerações humanas pode ser considerado verdade durante muito tempo antes ser identificado como falso. A humanidade hoje possui diversos métodos para validação do conhecimento e a procura por reafirmar fatos previamente estabelecidos certamente não é um deles. Ainda, diversos pontos de vista hoje confirmados como plausíveis foram, em seu tempo de proposição, tidos como inválidos.\par
Gonçalves é feliz em sua análise quando diz que o argumento de Gamaliel é falacioso, mas não pelos motivos apontados. Parece, na verdade, que é porque não observamos na história humana o domínio soberano da verdade e o declínio do erro. Tanto o erro pode ser propagado sem muitas dificuldades, como a verdade é facilmente enterrada e pisada pelo fluxo dos acontecimentos ou por interesses diversos.

\section*{E a Revelação Cósmica?}

Jan Val Ellam\cite{ref1}, escritor e palestrante que tem proposto novos tipos de análise a contextos históricos já conhecidos, bem como novos contextos a serem analisados, denomina suas colocações como pertencentes a um conjunto de conhecimentos chamado Revelação Cósmica (RC).\par
Pode-se dividir as informações fornecidas pela RC em \emph{constatáveis} e \emph{plausíveis}. Devo deixar claro, aqui, que essa divisão é feita exclusivamente por mim e para fins de esclarecimento pessoal, isto é, não passou por avaliação de nenhuma outra consciência perturbada a não ser a de quem vos escreve. \par
Primeiro, as informações \emph{constatáveis} são aquelas que qualquer um pode, por mérito e esforço próprio, concluir. Elas podem depender de mais ou menos conhecimentos específicos em certos domínios científicos como história, filosofia, biologia, neurociência, psicologia; e não científicos como esoterismo, espiritualidade, mitologia, ufologia, entre outros. É característica fundamental da RC essa interdisciplinaridade nas suas abordagens. A apreensão dessas informações, portanto, não depende de estarem envolvidas no corpo de uma "Revelação", porquanto podem ser adquiridas por qualquer buscador honesto e decente.\par
As \emph{plausíveis} são aquelas que mesmo que alguém queira, com a maior boa vontade e dedicação, não poderia chegar a ter. Essas dizem respeito a uma parte significativa da RC e que se apresenta na forma de relato. Ellam se encontraria, então, na posição de narrador da história e relataria - com a autoridade de quem alegadamente tem contato com diversas categorias de seres - aquilo que a massa não conseguiria enxergar por si só. Assim, essa parte pode ter no máximo a denominação de \emph{plausível} porque não pode ser verificada diretamente por mais ninguém, o que não impede que sejam razoáveis ou lógicas.\par
A combinação desses dois estilos de informação conferem à RC um poder persuasivo muito forte, sobretudo porque o que é \emph{plausível} facilmente se justifica no que é \emph{constatável}, e os relatos são de fácil credibilidade dada a honestidade e confiança que o locutor transmite. Ainda assim, é prudente e recomendado se lembrar com frequência de que não há motivos claros e óbvios para acreditar firmemente naquilo que não é constatável. Talvez essa postura tenha feito falta na percepção de Gamaliel. \par
É difícil falar sobre o tipo de informação as quais Gamaliel teve acesso ao tempo de sua vida. De fato, pelo espírito da época, era plausível falar sobre a soberania da obra de Deus no mundo, de modo que o que fosse contraditório a isso se desfizesse. Porém, se torna quase impossível imaginar sobre se ele tinha elementos suficientes para constatar essa proposição. Como membro importante e reverenciado pelos fariseus, dificilmente poderia, ainda que conseguisse constatar a incoerência, falar abertamente sobre isso sem causar escândalo.\par
Pelo modo como a RC descreve o funcionamento das coisas, podemos dizer que talvez o espírito de Gamaliel aprecie essa tentativa de ajuste. Às vezes é preciso que alguém "desdiga" o que foi dito anteriormente com o intuito de promover novas sequências e "desenrolares" de acontecimentos. Ou aquilo proposto lá atrás se verifica com o passar do tempo (dificilmente o caso aqui), ou precisa ser ressignificado para dar guarida a novas percepções.

\bibliography{resigamrefs.bib}

\end{document}